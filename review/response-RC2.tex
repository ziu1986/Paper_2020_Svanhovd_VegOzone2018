% Authors' response
\documentclass{scrartcl}
\usepackage[utf8]{inputenc}
\usepackage[english]{babel}
\usepackage{xcolor}
\usepackage{graphicx}
\usepackage{hyperref}
\usepackage{multirow}
\thispagestyle{empty}

\begin{document}
\section*{Authors' response}
To bg-2021-260-RC2 (23 Nov 2021):
We thank the anonymous referee \#2 for their very constructive comments and useful suggestions.
We will address the \textbf{major comments} raised regarding the structure and readability of the manuscript in our next revision. We will refine the language to reduce ambiguity and supplement our statements with appropriate citations.\\
As the manuscript will, hence, undergo major structural changes, we have to refrain from addressing all \textbf{specific comments} in detail at this point. In the final revision of the manuscript, we will take all specific and technical comments which are still relevant into consideration.

Here, we shall briefly respond to all issues where it seems appropriate.
\subsection*{General comments}
\begin{itemize}
    
    \item {\color{blue} "[...] main figures is missing [...]"} The referee is referring to Fig.~8 which is apparently absent from their version of the manuscript. We downloaded the pre-print from the associated discussion page (\href{https://bg.copernicus.org/preprints/bg-2021-260/}{bg-2021-260}) and cannot confirm Fig.~8 to be missing. It can be found on page~23 (just before the list of references). We, however, acknowledge that the current placement (after the main body of text) is rather unfortunate. This shall be fixed in the final typesetting process!

\end{itemize}

\subsection*{Specific comments}
\begin{itemize}
    
    \item {\color{blue} "[...] I find the introduction to be a bit longer than would be most effective for conveying this information. It would be helpful to break it into subsections [...]. [...]content of [...] lines 115--134 is important for contextualizing [...] could be better emphasized [...] or [...] opening paragraph for [...] section~2 [...]."} Thank you for this constructive comment. We will condense our introduction and put subsections in place where necessary. As suggested, we move the discussion of meteorology to Section~2.
    \item {\color{blue} "Please correct the text in line 105. [...] This doesn't seem to be a correct descriptor for Equation~2 [...]"} Thank you for pointing out this carelessness which should not have gone unnoticed.
    \item {\color{blue} "$f_\mathrm{T}$ hasn't been defined yet." "Is $f_\mathrm{T} = f_\mathrm{temp}$?"} These two issues are related. Of course, $f_\mathrm{T}$ is the same as $f_\mathrm{temp}$ and we should not have used both at the same time. We remove this ambiguity.
    \item {\color{blue} "Line 317-318 -- Mention that this is PPFD~0.8 ans 1.2 in parentheses [...]"} We add the suggested information to clarify our naming convention.
    \item {\color{blue} "Line 321 appears to substitute for $f_\mathrm{light}$" in Equation 4, not for as described in the text."} Indeed this is sentence ambiguous, and should read: "First, we solve Eq.~(3) for the MM default values of $\alpha_\mathrm{MM}$ at $f_\mathrm{light}=0.5$" We should, perhaps, clarify that we use $\gamma$ synonymously with PPFD in this paragraph or substitute $\gamma$ with PPFD.
    \item {\color{blue} "Line 331 -- Is the low standard deviation specific to nighttime?"} No, we have selected our sample (May--October), so that the vegetation will not experience "nighttime" between 5~am and 9~am. See, f.e. \href{https://www.timeanddate.com/sun/@777232}{TimeAndDate.com/Svanvik}.
    \item {\color{blue} "I find the conclusion that the standard deviation indicating "higher robustness to variability in growing conditions" to sweeping relative to the evidence precented. [...]"} This might be a misunderstanding due to inaccuracies in language? The robustness is not a conclusion but a presumption. We have not explicitly stated the associated standard deviation of our sampled essential climate variables which we could easily do to strengthen our point. We assume that the essential climate variables (excluding ozone) vary substantially in our 20 years time frame -- meaning having a large spread. A low standard deviation in stomatal conductance, therefore, suggests that stomatal conductance would to be relatively unaffected which we call "robustness". We will make this more clear.
    \item {\color{blue}}

\end{itemize}

\subsection*{Technical comments}
\begin{itemize}
    
    \item {\color{blue} d}

\end{itemize}


\end{document}