% Authors' response
\documentclass{scrartcl}
\usepackage[utf8]{inputenc}
\usepackage[english]{babel}
\usepackage{xcolor}
\usepackage{graphicx}
\usepackage{hyperref}
\usepackage{multirow}
\thispagestyle{empty}

\begin{document}
\section*{Authors' response}
To bg-2021-260-RC2 (23 Nov 2021):
We thank the anonymous referee \#2 for their very constructive comments and useful suggestions.
We will address all \textbf{general comments} raised regarding the structure and readability of the manuscript in our next revision. We will refine the language to reduce ambiguity and supplement our statements with appropriate citations where indicated by the referees.\\
As the manuscript will, hence, undergo major structural changes, we can only addressing some \textbf{specific comments} in detail at this point. In the final revision of the manuscript, we will take all specific and technical comments which are still relevant into consideration.\\
In the following, we shall respond to all issues where reasonable at this point.
\subsection*{General comments}
\begin{itemize}
    
    \item {\color{blue} "[...] main figure is missing [...]"} The referee is referring to Fig.~8 which is apparently absent from their version of the manuscript. We downloaded the pre-print from the associated discussion page (\href{https://bg.copernicus.org/preprints/bg-2021-260/}{bg-2021-260}) and cannot confirm Fig.~8 to be missing. It can be found on page~23 (just before the list of references). We, however, acknowledge that the current placement (after the main body of text) is rather unfortunate. This shall be fixed in the final typesetting process!

\end{itemize}

\subsection*{Specific comments}
\begin{itemize}
    
    \item {\color{blue} "[...] I find the introduction to be a bit longer than would be most effective for conveying this information. It would be helpful to break it into subsections [...]. [...]content of [...] lines 115--134 is important for contextualizing [...] could be better emphasized [...] or [...] opening paragraph for [...] section~2 [...]."} Thank you for this very constructive comment. We will condense our introduction and put subsections in place where necessary. As suggested, we move the discussion of meteorology to Section~2.
    \item {\color{blue}"Line 102 -- The concept of critical loads is much older than 2016 [...] please include an earlier citations [...]"} Indeed the concept is much older (1970s). Critical load is a term defined by the UNECE's Convention on Long Range Transboundary Air Pollution (LRTAP) and is generally associated with wet deposition of N and S. For ozone (and other pollutant gases (i.e. not in solution) we refer to 'critical levels'. Either concept is the level or load of the pollutant below which damage or injury would not be expected to occur to the receptor in question (e.g. forests, crops, grasslands etc...). We will include an appropriate citation.   
    \item {\color{blue} "Please correct the text in line 105. [...] This doesn't seem to be a correct descriptor for Equation~2 [...]"} Thank you for pointing out this carelessness which should not have gone unnoticed.
    \item {\color{blue} "In the paragraph that starts at line 241, the meaning of the word "significant" is unclear and varies within the paragraph. [...] it should be stated early on in this section. [...] tie back what is said to the objective of the paper."} Thank you. We will restructure the section to make the definition of the term "significance" easier assessable. The results shall be put into context with the paper's objective, either in-section or in a "results" Section.
    \item {\color{blue} "Section~4 - Please expand the description of the $\mathrm{DO_3SE}$ model"} A comprehensive model description can be found in Appendix~B. As no modifications to the $\mathrm{DO_3SE}$ model were made (except for the change in parameters), we had concluded that an excessive model description would distract the reader. In particular, relationships between plant physiological parameters and meteorological observations are described in Appendix~B1. Model "high-level" input and its preparation is described in Appendix~B2. We agree that important information is therefore missing from the main text. In the course of restructuring, we include a condensed model description in Section~4.
    \item {\color{blue} "$f_\mathrm{T}$ hasn't been defined yet." "Is $f_\mathrm{T} = f_\mathrm{temp}$?"} These two issues are related. Of course, $f_\mathrm{T}$ is the same as $f_\mathrm{temp}$ and we should not have used both at the same time. We remove this ambiguity.
    \item {\color{blue} "Line 304 (and Line 237-328) Again "extreme" is ambiguous here and seems to differ in definition from previous use [...]."} We will carefully evaluate the use of the term and define it appropriately.
    \item {\color{blue} "Lines 306-307 -- Please clarify which time frames are being compared related to what is meant by "subject to climate change". In other words, are you comparing 1990s to 2000s? Or 1990s + 2000s with implicit impacts of climate relative to the preindustrial era?"} In the context of our study, the time frame for climate change is 2010s compared to early 1990s.
    \item {\color{blue} "Line 317-318 -- Mention that this is PPFD~0.8 ans 1.2 in parentheses [...]"} We add the suggested information to clarify our naming convention.
    \item {\color{blue} "Line 321 appears to substitute for $f_\mathrm{light}$" in Equation 4, not for as described in the text."} Indeed this sentence is ambiguous, and should read: "First, we solve Eq.~(3) for the MM default values of $\alpha_\mathrm{MM}$ at $f_\mathrm{light}=0.5$" We should, perhaps, clarify that we use $\gamma$ synonymously with PPFD in this paragraph or substitute $\gamma$ with PPFD.
    \item {\color{blue} "Line 331 -- Is the low standard deviation specific to nighttime?"} No, we have selected our sample (May--October), so that the vegetation will not experience "nighttime" between 5~am and 9~am. See, f.e. \href{https://www.timeanddate.com/sun/@777232}{TimeAndDate.com/Svanvik}.
    \item {\color{blue} "I find the conclusion that the standard deviation indicating "higher robustness to variability in growing conditions" to sweeping relative to the evidence precented. [...]"} This might be a misunderstanding due to inaccuracies in language? The robustness is not a conclusion but a presumption. We assume that the essential climate variables (excluding ozone) vary substantially in our sampled 20 years. Though, we have not explicitly shown the associated standard deviation of this subsample of our data. We could do this to strengthen our point. A low standard deviation in stomatal conductance, therefore, suggests that stomatal conductance is relatively unaffected by these variations. We refer to this as "robustness". We will make this more clear in the next revision. Or remove the whole point as suggested.
    \item {\color{blue} "Line 332 -- "subarctic-PPFD0.8" is "best" relative to what?"} In this case "best" refers to the highest relative stomatal conductance and smallest standard deviation compared to the other probed combinations of parameter. We rephrase the sentence accordingly.
    \item {\color{blue}"Is there a citation or justification for the 1 nmol/m2/s flux threshold?"} This threshold (amongst others) has been introduced for modeling ozone induced damage on forest as part of the ICP Vegetation taskforce in the framework of LRTAP. "A uniform $\mathrm{O_3}$ flux threshold of $Y=1 \mathrm{nmol~m^{-2}~s^{-1}~PLA}$ was adopted for use in $\mathrm{POD_Y SPEC}$ for all tree species at the $\mathrm{O_3}$ Critical Levels workshop in Madrid, November 2016, based on data and analyses as presented in Büker et al. (2015). For the majority of tree species, this threshold fulfilled the recommendations that the confidence interval of the intercept includes 100\% and that the $R_2$ value is within 2\% of the maximum $R_2$ value (Büker et al., 2015)." (Mapping Manual; Mills et al. (2017)) We will include this in the revised manuscript.
    \item {\color{blue}"Figure~8 appears to be missing from the document."} As mentioned earlier, the placement of the figure on page 23 (after the main body of text) is unfortunate. The figure is not missing (at least as far as we can see), but might, therefore, have slipped the vigilant eye of the referee.
    \item {\color{blue}"Lines 367-372 -- "leads to" seems strong, given that there are many variables changing [...]"} We change the sentence accordingly.
    \item {\color{blue}"Where us ut shown that temperature acclimation relates to the amplification of drought effects?"} This is shown as part of Fig.~8 in which we summarize all sensitivity studies. Drought only affects the ozone dose in 2018 and not in 2019. The effect is stronger the more acclimated to cold temperatures our parameterizations are.

\end{itemize}

\subsection*{Technical comments}
We will appropriately address all comments raised, but only address selected items at this point.
\begin{itemize}
    \item {\color{blue} Concerning Figures 3--5} Thank you for the useful suggestions to further improve our figures. We will take them into consideration. We may kindly ask to review Figure~8 as well.
    \item {\color{blue} I haven't previously seen this method for identifying the start and end points of photosynthesis, and (noting that I'm not an ecologist) it seems elegant in concept and approach.} Thank you. This was a pragmatic, data-driven approach. To our knowledge, most satellite data based studies on start and end of growing season focus on correlations with essential climate variables to identify processes and relationships.
    \item {\color{blue}"Sentence in lines 339--340: "This value will be used for all PFTs alike". Please specify -- $\mathrm{A_{end}}$? In this paragraph please explain why it is acceptable to use the same $\mathrm{A_{end}}$ for all PFTs while $\mathrm{A_{start}}$ is specific."} It is still difficult to predict when deciduous vegetation will start the abscission process. For practical reasons, we set the date for all PFTs alike, although it could differ in reality. Coniferous trees are capable of photosynthesis as long as there is light which is the hard limit for all species alike. Grassland, on the other hand, will in most cases stop photosynthesis after it has been covered by snow.
    \item {\color{blue} "[...] please clarify what leaf length you are using [...]"} We have conducted two sets of simulations, one with leaf dimensions, tree height, and growing season as defined in the mapping manual and one with leaf dimensions of deciduous trees adjusted to the measured size. We have not adjusted the leaf dimensions of coniferous trees and grassland. We will make this clearer.
    \item {\color{blue} "Line 457 -- do you mean the response to SWP, rather SWP itself being negligible?"} Of course! We apologize for the sloppy language.

\end{itemize}


\end{document}