% Authors' response
\documentclass{scrartcl}
\usepackage[utf8]{inputenc}
\usepackage[english]{babel}
\usepackage{xcolor}
\usepackage{graphicx}
\usepackage{hyperref}
\usepackage{multirow}
\thispagestyle{empty}

\begin{document}
\section{Final response}
In response to the major concerns of the referees, we had to rephrase our research question. We now focus on the uncertainty in the assessment of ozone-induced damage risk that follow the UNECE LRTAP Convention concerning the subarctic region. In these assessments, it is assumed that Boreal parameterizations of stomatal conductance are sufficient to describe also the response of subarctic vegetation.

To this end, we propose a local climate-derived temperature parameterization of stomatal conductance by shifting the existing Boreal parameterizations to lower temperatures. This is analogous to what has been experimentally found for temperature acclimation of $T_\mathrm{opt}$ in plant photosynthesis. We adjust stomatal conductance because the used model doesn't compute photosynthesis.

Based on this, we restructured the manuscript taking all relevant referee comments into consideration. 
\begin{itemize}
    \item The introduction has been condensed.
    \item Speculative phrases and formulations removed or rephrased.
    \item We separated methods, results, and discussion into respective sections.
    \item The discussion about local climate was also condensed and put into the context of the research question.
    \item Considering the attribution of "significance", we refrain now from using this term inappropriately and
    \item reworked the statistical analysis and presentation of the environmental data.
    \item We use box plots to condense statistical information about the data distribution.
\end{itemize}
      
Considering the comparison with flux measurements, we were not able to find appropriate data for northern Fennoscandia. Therefore, we discuss the difference in modeled ozone dry deposition velocities for coniferous trees with what has been observed at the forest research station Hyytilälä in Finland that is further south and not applicable to deciduous forests or perennial grassland. The comparison showed good agreement in summer but probably too high modeled deposition in spring.

\end{document}