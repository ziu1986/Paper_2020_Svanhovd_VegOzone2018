% Authors' response
\documentclass{scrartcl}
\usepackage[utf8]{inputenc}
\usepackage[english]{babel}
\usepackage{xcolor}
\usepackage{graphicx}
\usepackage{hyperref}
\usepackage{multirow}
\thispagestyle{empty}

\begin{document}
\section*{Authors' response}
To bg-2021-260-RC3 (30 Nov 2021):
We thank the anonymous referee \#3 for their comments.
We will address all \textbf{general comments} raised regarding the structure and readability of the manuscript in our next revision or resubmission. We consider separating the manuscript in Introduction--Methods--Results--Conclusions and will improve on the "methods" Section to make our results more comprehensible. We will refine the language to reduce ambiguity and supplement our statements with appropriate citations where those were indicated by the referee.\\
As the manuscript will, hence, undergo major structural changes, we will not address all \textbf{specific / technical comments} in detail at this point. We will take all specific and technical comments which are still relevant into consideration.\\
In the following, we shall give a brief response to all relevant issues.

\subsection*{Major comments} 
\begin{itemize}
    
    \item {\color{blue}Concerning the length of the paper, I really cannot see the benefit of all the work on long-term ozone and meteorology given in Sections 2-3. Isn't the main point of this paper to illustrate differences in vegetation characteristics between plants at nearly 70 degrees N and the Mapping Manual (MM)? That doesn't need a long analysis of climate differences or ozone trends.}
    You are right and upon restructuring the manuscript, we will address this matter by condensing this part. We deemed a detailed analysis of long-term ozone and meteorological conditions important for modeling and understanding past, present, and future ozone risk in the subarctic. Our specific sample period extends only over two growing seasons (2018/2019). To assess whether or not any of the observed visible damage in 2018 could correspond to an unusual high ozone dose, we had to set this particular year into a climatological context. 
    
    \item {\color{blue}Concerning the "default" growing seasons (from DOY 100 to 307) from the MM against which the "bespoke" are compared, these are just wrong for this latitude. The MM suggests a latitude function that gives DOY 129 to 268.}
    In retrospect, this was a fault in the model setup and will be corrected. However, we clearly state in L344, that these start and end DOYs for grassland are valid only to central Europe.
    
    \item {\color{blue}Some of the basic equations look wrong to me, see my comments on Appendix B below.}
    We will double-check the equations (see comments below) and make the description of the $\mathrm{DO_3SE}$ model more comprehensible and comprehensive. However, in the interest of the main focus of this manuscript, it may not be possible to give a full recap of all details concerning this well-established model.
    
    \item {\color{blue}On p21 the author's first "key findings" state that their "bespoke parameterizations for subarctic species are better suited..." than the MM ones. This is a strong statement given that the authors admit on L293 that their bespoke suggestions are "hypothetical and have yet to be verified by observations".}
    Thank you for pointing this out. We will rephrase our key findings accordingly.
\end{itemize}

\subsection*{Other comments} 
\begin{itemize}

\item {\color{blue}L14. one can ask if an underestimate of 6\% is a significant problem. I would guess that the overall uncertainty of POD calculations is far greater than this!}
To address the underlying uncertainty caused by the choice of parameterizations that are used to compute POD, we varied the temperature and light response paramterizations (accounting for the right start and end of the growing season) and looked at the influence of SWP on POD in our target region. Within this parameter space, the uncertainty due to temperature and light response is, e.g., $\sim3\,\%$ for deciduous trees. (For which we found the $6\,\%$ underestimate.)
From Table 4 we see that the deciduous tree biomass loss is $11.2\,\%$ with MM settings and $17.4\,\%$ with bespoke settings for temperature and GS. The difference is 6 percentage points, but the relative increase between losing $11\,\%$ and $17.4\,\%$ is $55\,\%$. Hence, we conclude that the estimated underestimate is significant (albeit not in a strict statistical sense). We will add a discussion about the methodological uncertainty of POD in the next revision and may rephrase the sentence.

\item {\color{blue}L22. Better to say "estimated" rather than reported. These numbers do not represent actual measured yield losses. Also, as this paper is mainly discussing DO3SE-type approaches to POD, it would have been relevant to cite the DO3SE-based estimates of the Mills et al 2018 paper (below) which also made yield estimates for four crops.}
Thank you for pointing this out. We will rephrase and include the relevant citations.

\item {\color{blue}L29. A more recent citation would be Tarasick et al 2019.}
Thank you. We will include the citation.

\item {\color{blue}L36. Are these numbers for tropospheric ozone relevant to this paper? Vegetation is affected by near-surface ozone, and levels of this are much lower than 50-65 ppb over the Nordic countries.}
These numbers are indeed not relevant for the monitoring station at Svanhovd which is situated at a low altitude. We will remove the reference.

\item {\color{blue}L39-40. Why "despite"? 22 days is a very long time! This time-scale is not so relevant though; it is the lifetime of ozone in the boundary layer that matters, and that is about 1 week (Hov et al., 1978, or as can be estimated by typical deposition velocities and boundary layer depths). Solberg et al. 2008 also show how residence times of less than 7 days are usual for ozone episodes.}
Thank you for pointing this out. We will rephrase the sentence to include the more relevant time scales. When citing 22 days, we had the long-range transport (trans-continental) in mind.

\item {\color{blue}L48. Spring peaks are seen across large swathes of the northern hemisphere, for example at Mace Head, caused by many factors (Monks et al., 2009). In fact, Monks et al cite Winkler (1988) that north of 60N, the maximum in ozone reduces in amplitude and shifts to later months.}
Thank you for pointing this out. We will rephrase the sentence accordingly.

\item {\color{blue}L85. I don't think Simpson et al 2007 or Mills et al 2011, 2017 focused on the Mediterranean? These were pan-European studies, that also stressed that POD as a metric showed that risks in e.g. northern Europe were likely higher than the earlier AOT40 metrics showed. The ICP-vegetation type metrics have in fact been based upon data from countries across Europe, and with more data from central and northern than from southern European countries.}
We will carefully check the cited works and rephrase the paragraph accordingly. 

\item {\color{blue}L89. Continuing the above, the studies from Karlsson et al have had a strong focus on Scandinavian vegetation.}
Our focus lies in particular on vegetation in the subarctic. Whereas works by Karlsson et al. focused mainly on vegetation further south in the boreal zone. 

\item {\color{blue}L96. Define PLA}
PLA stands for "projected leave area" indicating that the area is measured on only one side of the leaf. It also means that any two-sided effects of leaves need to be accounted for by doubling the leaf area in calculations.

\item {\color{blue}L97. I don't think you need this about typically exceeded under daylight hours. POD calculations follow DO3SE methods, and the flux is zero at night by definition.}
We will remove it accordingly.

\item {\color{blue}L103. The Maas et al book isn't so easily available (and no web-access address is given anyway). It can be good to cite some of the other key papers behind ozone CLs, e.g. Fuhrer et al. (1997), Mills et al 2011 (below).}
We will include the given references in addition.

\item {\color{blue}L105. "The CL is calculated by"? Do you mean exceedance of CL?}
Of course, we mean exceedance.

\item {\color{blue}L110. "For our study...". This confused me. Are the authors reporting results from the present manuscript, or from some unpublished and uncited study? The sentences starting here seem to be out of place.}
The ozone monitoring at Svanhovd was originally meant as a supplement for another study in our project. Hence, the confusing formulation. We will rephrase the sentence.

\item {\color{blue}L127. Do you mean non-methane VOC (NMVOC)?.} 
Yes, we mean non-methane VOCs.

\item {\color{blue}L128. Why brackets in "[O3]"?}
The "$[X]$" notation commonly refers to the concentration of a chemical species X. Parts-per-billion (ppb) is a volume mixing ratio (VMR) and strictly speaking no concentration. Though, if you calculate the amount of ozone per volume from ppb to $\mathrm{\mu g\,m^{-3}}$ meter, assuming a pressure of 1 atmosphere, the temperature of 298K and use the ideal gas law, you get a factor $\sim 2$. Therefore, concentrations and VMRs are often used synonymously. \\We could also use another terminology, e.g. $\chi_{O_3}$ (\href{https://www.e-education.psu.edu/meteo300/node/534}{e-education.psu.edu}). \\The "$\left<A\right>$" notation is derived from Dirac's "bra-ket" formalism in quantum mechanics, e.g. $\left<\Psi |A|\Psi \right>\rightarrow \left<A\right>$ "this expression gives the expectation value, or mean or average value, of the observable represented by operator A for the physical system in the state $\left|\Psi \right>$ " (\href{https://en.wikipedia.org/wiki/Bra%E2%80%93ket_notation}{wikipedia}).

\item {\color{blue}L131-132. I didn't quite understand "shall serve as a reference". Do you mean as an example or possible future?}
Exactly. "Example" may be the right term in this context. 

\item {\color{blue}L141. Why "sensu" here. I had to look up the word, and read that it is "used especially in technical taxonomic references". Use simpler English.}
We will rephrase.

\item {\color{blue}L141. In any case what does "(sensu World Meteorological Organisation)" mean in this sentence?}
In general, it means "in the sense of". But in this context, it could also be read as "refer to" for further reading. "Essential Climate Variables" is a term defined and commonly used by the WMO that might not be well known in the plant physiology community.

\item {\color{blue}L165-167. Mangled sentences? Why "Last accessed"? There is no url here. Placement of (a) and (b) is strange.}
This was supposed to be in the bibliography. We will check the bibtex citation styles to fix the (a) and (b) placements.

\item {\color{blue}L170-171. Why deal with data that may introduce a false trend at all? Why wouldn't the problems influence seasonality? I would guess that weather and climate conditions may well have influenced the frequency of quality-assurance checks.\\
L172. Here it says that agromet variables are available from 1992, so, again, why use problematic O3 data from the 1980s?}
These two questions are related. Re-Consulting the ozone monitoring report by Solberg et al. (2003), we find an inaccuracy in our manuscript text which should read "[...] from before 1997 [...]". The systematic uncertainty in the pre-1997 ozone date is of the order of $10\,\%$ in AOT. This means that the uncertainty in each measurement is lower but in the order of the trend in background ozone. This concerns the whole historical monitoring period at Svanhovd. Combining all data gives good confidence in the observed seasonality rather than each measurement. From this, we find, that 2019 was a rather usual year in terms of ozone and weather. Without relying on ozone climatology, we could not make this assessment. In \href{https://acp.copernicus.org/articles/21/15647/2021/}{Falk et al. (2021)} we have shown that the correlation between data taken at Svanvik and Pallas in Finnland during the overlapping period in the 90s is $r^2=0.61$ which is good given the different altitudes and surroundings.

\item {\color{blue}L214, Fig. 3. The stars are very small on the figure. Make these more obvious. Or just skip them - doesn't the temperature curve give enough information?}
We will revise the figure accordingly.

\item {\color{blue}L215-224. Do we need to be told all of these details with plus/minus limits? These are just climatalogical values and can be seen well enough in Fig. 3. The numbers are not used for anything.}
We consider removing these in the next revision.

\item {\color{blue}L255-256. Why "deduce"? Now the numbers given, 800 W/m2 and 200 W/m2, are very rounded - are these measurements or assumptions or limits?}
Do we assume the referee refers to L225-226? We agree that "deduce" is not the right term in this context and will rephrase. These are limits based on the observations.

\item {\color{blue}L276. I would say hypothetical or local rather than bespoke.}
Thank you for pointing this out. We will change the term accordingly.

\item {\color{blue}L280. Again, why "deduce"?}
See above.

\item {\color{blue}L281-282. The better ref here for the generic PFTs is the 2017 Mapping Manual. Or what did you use from Simpson et al 2007 or Mills et al. 2011 that isn't in Mills et al 2017?}
This was probably a copy and paste mistake as all values here were taken from the MM except for LAI.

\item {\color{blue}L286. First, what is Gstoleaf compared to gsto? On L553 it is stated that Gsto(leaf) is at leaf-level, which suggests that same as gsto in eqn (B1). Then, you claim that Fig. B1 shows low Gsto(leaf), but it shows enormous values! According to Fig. B1, Gstoleaf is ca. 100 mmole/m2/s, this is 1.0e5 nmole/m2/s! This makes no sense.}
$g_\mathrm{sto}$ is the stomatal conductance per projected leaf area while $G_\mathrm{sto}$ is per $\mathrm{m}^2$. Hence, $G_\mathrm{sto}$ at leaf level is $g_\mathrm{sto}$ multiplied by the leaf area of the whole plant.

\item {\color{blue}L292. The authors say that their fT system has not been verified, but haven't ecosystem models (e.g. JULES, CLM, LPJ-GUESS) parameterized such vegetation already? I would have thought that there was something to learn from such models.}
Thanks for pointing this out. It depends on what type of $g_\mathrm{sto}$ module ecosystems models use. Generally, these will use coupled photosynthesis-stomatal conductance models which do not require the definition of an $f_\mathrm{T}$ function.

\item {\color{blue}L344. As noted above, why compare your Astart (129, 130) with the MM value of 100 which is valid for 50 degrees N? The MM suggests a latitude function, which gives a start date of 129 for your location - i.e. almost perfect, and your comment was misleading. Why would you consider an agricultural criteria for forests anyway? Why not apply the MODIS response for deciduous trees too?}
The agricultural rule refers to the bud burst of birch trees. For a thorough discussion of the use of accumulated mean temperatures in estimating the start and end of the growing season in Scandinavia see e.g. \href{https://pub.epsilon.slu.se/3910/1/SFS194.pdf}{Morén and Perttu (1994)}. We will make this clearer. \\The choice of $A_\mathrm{start}$ for a central European latitude is indeed misleading. This was a fault in the model setup and will be corrected.

\item {\color{blue}Table 3: Again, the "Defaults" given here are incorrect for this latitude. The correct ref should anyway be the MM.}
See above.

\item {\color{blue}L345. "Due to the lack of quantitative field observation...". I didn't follow the logic here.}
We qualitatively checked the start of the GS by looking at the live stream from a nearby webcam. This cannot be used for quantitative analysis. Hence, snow depth measured at a meteorological station is taken into consideration instead. From this observation, we inferred a latency period of about one month after snowmelt for the grass to sprout. We will rephrase this sentence to make this clearer.

\item {\color{blue}Table 4. Why average the MM but not the bespoke? Give the MM for both years - this is useful information.}
Thanks for pointing this out. We will change the table accordingly.

\item {\color{blue}L488. Give references for your statements that global land-surface models have problems in the Arctic regions.}
We will include references in the next revision.

\item {\color{blue}L624-625. Mangled ref. And why source googlebooks, when this report is easily available from www.emep.int?}
Thanks. We will check our list of references.

\item {\color{blue}References: check upper-lower case, in e.g. AMBIO, grennfelt etc.}
Thanks. We will check our list of references.
 
\item {\color{blue}Appendix B: DO3SE model
The equations used here, and especially around L524-532, have several problems:
\begin{itemize}
\item they do not account for differences between canopy and leaf-scale conductances
\item they make no mention of how the ozone is calculated at the height of the vegetation from the 2m measured values.
{\color{black}The $\mathrm{DO_3SE}$ model was used to estimate the difference in $[\mathrm{O_3}]$ between a reference height above the canopy and the canopy height. This employs the deposition component of the $\mathrm{DO_3SE}$ model that estimates the transfer of mass (i.e. ozone) as a function of wind speed, convection, surface roughness, and vegetation $[\mathrm{O_3}]$ sink. We will add additional detail in the paper to make this clear.}
\item Equation B6 looks wrong. What is the u(z1) term doing here?
{\color{black}We will double check the equations.}
\item What is "z1" in eqn. B7?
{\color{black}This comes from the $\mathrm{DO_3SE}$ model - we will add a suitable reference.}
\item L531 the factor 1.3 accounts for the "difference in" diffusivity.
{\color{black}We will rephrase this sentence.}
\item Figure B2 - does MODIS only respond to coniferous trees in this 1x1 km2 grid? No other vegetation?
{\color{black}Estimated land use (from areal pictures available at kartverket.no): 48\% grassland, 32.5\% deciduous, 12.5\% coniferous, 2\% urban. We will include this information in the figure caption.}
\end{itemize}}
 
\end{itemize}

\end{document}