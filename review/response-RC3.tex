% Authors' response
\documentclass{scrartcl}
\usepackage[utf8]{inputenc}
\usepackage[english]{babel}
\usepackage{xcolor}
\usepackage{graphicx}
\usepackage{hyperref}
\usepackage{multirow}
\thispagestyle{empty}

\begin{document}
\section*{Authors' response}
To bg-2021-260-RC3 (30 Nov 2021):
We thank the anonymous referee \#3 for their comments.
We will address all \textbf{general comments} raised regarding the structure and readability of the manuscript in our next revision or resubmission. We consider to separate the manuscript in Introduction--Methods--Results--Conclusions and will improve on the "methods" Section to make our results more comprehensible. We will refine the language to reduce ambiguity and supplement our statements with appropriate citations where indicated by the referee.\\
As the manuscript will, hence, undergo major structural changes, we will not address all \textbf{specific / technical comments} in detail at this point. We will take all specific and technical comments which are still relevant into consideration.\\
In the following, we shall give a brief respond to all relevant issues.

\subsection*{Major comments} 
\begin{itemize}
    
    \item {\color{blue}Concerning the length of the paper, I really cannot see the benefit of all the work on long-term ozone and meteorology given in Sections 2-3. Isn't the main point of this paper to illustrate differences in vegetation characteristics between plants at nearly 70 degrees N and the Mapping Manual (MM)? That doesn't need a long analysis of climate differences or ozone trends.} A detailed analysis of long-term ozone and meteorological conditions is important for to modeling of ozone risk in the subarctic. Our sample period only extends over two growing seasons (2018/2019). To assess wether or not the visible damage could correspond to an unusual high ozone dose in 2018, we need to set this particular year into a climatological context. Upon restructuring the manuscript, we may remove or condense this part.
    
    \item {\color{blue}Concerning the "default" growing seasons (from DOY 100 to 307) from the MM against which the "bespoke" are compared, these are just wrong for this latitude. The MM suggests a latitude function that gives DOY 129 to 268.} The referee is right regarding the latitude model at $70\,^\circ$N and we could have used these instead from the start. However, as we state in L344, these start and end DOYs for grassland refer to central Europe.
    
    \item {\color{blue}Some of the basic equations look wrong to me, see my comments on Appendix B below.} We will double check the equations make the description of the $\mathrm{DO_3SE}$ model more comprehensible and comprehensive. However, in the interest of the main focus of this manuscript it may not be possible to give a full recap of all details concerning this well established model.
    
    \item {\color{blue}On p21 the author's first "key findings" state that their "bespoke parameterizations for subarctic species are better suited..." than the MM ones. This is a strong statement given that the authors admit on L293 that their bespoke suggestions are "hypothetical and have yet to be verified by observations".} Thank you for pointing this out. We will rephrase our key findings accordingly.

\end{itemize}

\subsection*{Other comments} 
\begin{itemize}

\item {\color{blue}L14. one can ask if an underestimate of 6\% is a significant problem. I would guess that the overall uncertainty of POD calculations is far greater than this!}

\item {\color{blue}L22. Better to say "estimated" rather than reported. These numbers do not represent actual measured yield losses. Also, as this paper is mainly discussing DO3SE-type approaches to POD, it would have been relevant to cite the DO3SE-based estimates of the Mills et al 2018 paper (below) which also made yield estimates for four crops.}

\item {\color{blue}L29. A more recent citation would be Tarasick et al 2019.}

\item {\color{blue}L36. Are these numbers for tropospheric ozone relevant to this paper? Vegetation is affected by near-surface ozone, and levels of this are much lower than 50-65 ppb over the Nordic countries.}

\item {\color{blue}L39-40. Why "despite"? 22 days is a very long time! This time-scale is not so relevant though; it is the lifetime of ozone in the boundary layer that matters, and that is about 1 week (Hov et al., 1978, or as can be estimated by typical deposition velocities and boundary layer depths). Solberg et al. 2008 also show how residence times of less than 7 days are usual for ozone episodes.}

\item {\color{blue}L48. Spring peaks are seen across large swathes of the northern hemisphere, for example at Mace Head, caused by many factors (Monks et al., 2009). In fact, Monks et al cite Winkler (1988) that north of 60N, the maximum in ozone reduces in amplitude and shifts to later months.}

\item {\color{blue}L85. I don't think Simpson et al 2007 or Mills et al 2011, 2017 focused on the Mediterranean? These were pan-European studies, that also stressed that POD as a metric showed that risks in e.g. northern Europe were likely higher than the earlier AOT40 metrics showed. The ICP-vegetation type metrics have in fact been based upon data from countries across Europe, and with more data from central and northern than from southern European countries.}

\item {\color{blue}L89. Continuing the above, the studies from Karlsson et al have had a strong focus on Scandinavian vegetation.}

\item {\color{blue}L96. Define PLA}

\item {\color{blue}L97. I don't think you need this about typically exceeded under daylight hours. POD calculations follow DO3SE methods, and the flux is zero at night by definition.}

\item {\color{blue}L103. The Maas et al book isn't so easily available (and no web-access address is given anyway). It can be good to cite some of the other key papers behind ozone CLs, e.g. Fuhrer et al. (1997), Mills et al 2011 (below).}

\item {\color{blue}L105. "The CL is calculated by"? Do you mean exceedance of CL?}

\item {\color{blue}L110. "For our study...". This confused me. Are the authors reporting results from the present manuscript, or from some unpublished and uncited study? The sentences starting here seem to be out of place.}

\item {\color{blue}L127. Do you mean non-methane VOC (NMVOC)?.}

\item {\color{blue}L128. Why brackets in "[O3]"?}

\item {\color{blue}L131-132. I didn't quite understand "shall serve as a reference". Do you mean as an example or possible future?}

\item {\color{blue}L141. Why "sensu" here. I had to look up the word, and read that it is "used especially in technical taxonomic references". Use simpler English.}

\item {\color{blue}L141. In any case what does "(sensu World Meteorological Organisation)" mean in this sentence?}

\item {\color{blue}L165-167. Mangled sentences? Why "Last accessed"? There is no url here. Placement of (a) and (b) is strange.}

\item {\color{blue}L170-171. Why deal with data that may introduce a false trend at all? Why wouldn't the problems influence seasonality? I would guess that weather and climate conditions may well have influenced the frequency of quality-assurance checks.}

\item {\color{blue}L172. Here it says that agromet variables are available from 1992, so, again, why use problematic O3 data from the 1980s?}

\item {\color{blue}L214, Fig. 3. The stars are very small on the figure. Make these more obvious. Or just skip them - doesn't the temperature curve give enough information?}

\item {\color{blue}L215-224. Do we need to be told all of these details with plus/minus limits? These are just climatalogical values and can be seen well enough in Fig. 3. The numbers are not used for anything.}

\item {\color{blue}L255-256. Why "deduce"? Now the numbers given, 800 W/m2 and 200 W/m2, are very rounded - are these measurements or assumptions or limits?
L276. I would say hypothetical or local rather than bespoke.}

\item {\color{blue}L280. Again, why "deduce"?}

\item {\color{blue}L281-282. The better ref here for the generic PFTs is the 2017 Mapping Manual. Or what did you use from Simpson et al 2007 or Mills et al. 2011 that isn't in Mills et al 2017?}

\item {\color{blue}L286. First, what is Gstoleaf compared to gsto? On L553 it is stated that Gsto(leaf) is at leaf-level, which suggests that same as gsto in eqn (B1). Then, you claim that Fig. B1 shows low Gsto(leaf), but it shows enormous values! According to Fig. B1, Gstoleaf is ca. 100 mmole/m2/s, this is 1.0e5 nmole/m2/s! This makes no sense.}

\item {\color{blue}L292. The authors say that their fT system has not been verified, but haven't ecosystem models (e.g. JULES, CLM, LPJ-GUESS) parameterized such vegetation already? I would have thought that there was something to learn from such models.}

\item {\color{blue}L344. As noted above, why compare your Astart (129, 130) with the MM value of 100 which is valid for 50 degrees N? The MM suggests a latitude function, which gives a start date of 129 for your location - i.e. almost perfect, and your comment was misleading. Why would you consider an agricultural criteria for forests anyway? Why not apply the MODIS response for deciduous trees too?}

\item {\color{blue}Table 3: Again, the "Defaults" given here are incorrect for this latitude. The correct ref should anyway be the MM.}

\item {\color{blue}L345. "Due to the lack of quantitative field observation...". I didn't follow the logic here. Table 4. Why average the MM but not the bespoke? Give the MM for both years - this is useful information.}

\item {\color{blue}L488. Give references for your statements that global land-surface models have problems in the Arctic regions.}

\item {\color{blue}L624-625. Mangled ref. And why source googlebooks, when this report is easily available from www.emep.int?}

\item {\color{blue}References: check upper-lower case, in e.g. AMBIO, grennfelt etc.}
 
\item {\color{blue}Appendix B: DO3SE model
The equations used here, and especially around L524-532, have several problems:
\begin{itemize}
\item they do not account for differences between canopy and leaf-scale conductances
\item they make no mention of how the ozone is calculated at the height of the vegetation from the 2m measured values.
\item Equation B6 looks wrong. What is the u(z1) term doing here?
\item What is "z1" in eqn. B7?
\item L531 the factor 1.3 accounts for the "difference in" diffusivity.
\item Figure B2 - does MODIS only respond to coniferous trees in this 1x1 km2 grid? No other vegetation?
\end{itemize}}
  

\end{itemize}

\end{document}